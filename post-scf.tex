\begin{frame}
  \frametitle{Post-SCF methods}

  \textbf{Complications when going beyond SCF:}

  \vspace{2mm}

  \begin{itemize}
    \item New formalism in \emph{first quantization} developed by J.Kottmann and F. Bischoff
    \item CC2 residual equations:
      \begin{align}
      |\tau_i \rangle &= -2\Helm_i\left[
                         |\potential_{\tau_i}\rangle
                         + \nuclear |\tau_i\rangle
                         + 2\coulomb |\tau_i\rangle
                         - \exchange|\tau_i\rangle
                         \right]\\
      |u_{ij} \rangle &= -2\Helm_{ij}\left[
                         Q_{12}^t\tilde{g}_{12}^{ij} |t_it_j \rangle
                         + \nuclear |u_{ij}\rangle
                         + 2\coulomb_{12} |u_{ij}\rangle
                         - \exchange_{12}|u_{ij}\rangle
                         \right]
      \end{align}
    \item Singles: $|\tau_i\rangle$ is a 3D function and $\Helm_i$ is the usual (6D) BSH operator
    \item Doubles: $|u_{ij}\rangle$ is a 6D function and $\Helm_{ij}$ is a 12D BSH operator
    \item CC3/CCSD(T) would require 9D functions and 18D BSH operators!
    \item Key ingredients: SVD and \emph{regularizations} to remove cusps
    \item Bonus feature: reduced formal scaling, $O(N_{occ}^3)$ vs $O(N^5)$ for LCAO (CC2/MP2)
  \end{itemize}
  
  \vspace{5mm}
  \centering
  \tiny
  J. Kottmann, F. Bischoff,
  {\it J. Chem. Theory. Comput.},
  \textbf{13 (12)},
  5945-5955 (2017)\\
  J. Kottmann, F. Bischoff,
  {\it J. Chem. Theory. Comput.},
  \textbf{13 (12)},
  5956-5965 (2017)
\end{frame}
