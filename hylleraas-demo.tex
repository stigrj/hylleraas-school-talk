\documentclass[8pt]{beamer} % Note option
\usepackage[utf8]{inputenc}


% Select the hylleraas theme. Note the options.
\usetheme{hylleraas}


\begin{document}

\title{Official Hylleraas Centre Beamer Template}
\date{Presented for your pleasure}
\author{S.~Kvaal and A.~Novoselov}

\begin{frame}
  \titlepage
\end{frame}


\title{Hylleraas Centre for Quantum Molecular Sciences}
\date[ISPN ’80]{Presentation at the great symposium}
\author[Helgaker]{Trygve Helgaker and Kenneth Ruud}


\begin{frame}[dark]
  \titlepage
\end{frame}


\begin{frame}[image 2]
  \titlepage
\end{frame}

\begin{frame}[image 3]
  \titlepage
\end{frame}

\begin{frame}[image 4]
  \titlepage
\end{frame}

\title{Adding logos to title pages}
\addtitlepagelogo{\includegraphics[height=1.5cm]{erclogo-white.png}\hspace{.5cm}}
\addtitlepagelogo{\includegraphics[height=1.5cm]{sfflogo-white.png}}
\addtitlepagelogo{\includegraphics[height=1.5cm]{uitlogo-white.png}}
\addtitlepagelogo{\includegraphics[height=1.5cm]{uiologo-white.png}}
\begin{frame}[image 1]
  \titlepage
\end{frame}


\begin{frame}
  \frametitle{Outline}
  \tableofcontents
\end{frame}

\section{Introduction}

\begin{frame}
  \frametitle{Welcome to the Hylleraas template}
  \begin{columns}
    \begin{column}{0.5\textwidth}
      \begin{itemize}[<+->]
      \item This is a demo of the Hylleraas \texttt{beamer} template.
      \item It is coded by Simen Kvaal, based on the PowerPoint template by Alexey Novoselov.
      \item It uses the \alert{Formular typeface}, and for this to work, therefore requires the use of \alert{\texttt{xelatex}}, a version of \texttt{pdflatex} that allows system fonts to be used.
      \item This presentation is compiled with:\\ \texttt{\$ xelatex hylleraas-demo.tex}.
      \item You can also use \texttt{pdflatex}, which will use Helvetica instead.
      \item The template is a work in progress, and the user is encouraged to submit comments, feature requests and \alert{bug reports}.
      \end{itemize}
    \end{column}
    \begin{column}{0.5\textwidth}
    \begin{center}
      \begin{measuredfigure}
        \includegraphics[height=5cm]{Hylleraas-portrett.jpg}
        \caption{This is Egil Hylleraas}
      \end{measuredfigure}
    \end{center}
    \end{column}
  \end{columns}
\end{frame}


\begin{frame}{Fonts}

  \begin{itemize}
  \item  Main text: Formular. Licensed from Brown Fox to Hylleraas.
  \item Greek mathematics: Fira Sans from Google. Free.
  \item Install fonts using the system-provided tool on Mac, Windows or Linux systems. The font is then avalilable for \texttt{xelatex}.
  \item In a later version of the template, I will implemenet fallback to system fonts if standard \texttt{pdflatex} is run, or if Formular/Fira Sans are not available.
  \end{itemize}


\end{frame}

\begin{frame}
\frametitle{To be implemented}
\begin{enumerate}
\item
  Proper error handling for fonts
\item
  Fallback to Helvetica in the case of \texttt{pdflatex} and/or missing fonts
\item
  The current version is 4:3 only. Support for 16:9 will be added.
\end{enumerate}
\end{frame}

\section{Lists and enumerations}

\begin{frame}
  \frametitle{Itemize}
  \begin{itemize}
    \item This is an item
    \item Here is another item with some \alert{alerted text}
    \begin{itemize}
    \item Nested itemize
    \item It looks nice
    \end{itemize}
    \item A final item
  \end{itemize}
\end{frame}

\begin{frame}
  \frametitle{Enumerate}
  \begin{enumerate}
    \item This is an item
    \item Here is another item with some \alert{alerted text}
    \begin{enumerate}
    \item Nested enumerate
    \item It looks nice
    \end{enumerate}
    \item A final item
  \end{enumerate}
\end{frame}


\begin{frame}
  \frametitle{Description}
  Here is the \alert{\texttt{description}} environment:
  \begin{description}
    \item[Zeroeth law] Thermal equilibrium
    \item[First law] Conservation of energy, etc
    \item[Second law] Monotone entropy
    \item[Third law] Entropy goes to zero as $T\to 0$
  \end{description}
\end{frame}

\section{Mathematics}


\begin{frame}\frametitle{The mathematics fonts}
  \begin{itemize}
    \item Mathematics with sans serif fonts like Helvetica and Formular can be tricky
    \item I have used various tricks to make it look good enough for government work
    \item In the next slide we can see some examples
  \end{itemize}
\end{frame}

\begin{frame}{The look of mathematics in the Hylleraas template}

A few elements are displayed here:
\begin{itemize}
\item Greek letters: $\alpha$, $\beta$, $\gamma$, $\delta$, $\rho$, $\varrho$, $\sigma$, $\varsigma$, $\phi$, $\varphi$, $\psi$
\item Some uppercase Greek letters: $\Gamma$, $\Delta$, $\Psi$, $\Phi$, $\Theta$
\item This is the time-independent Schrödinger equation:
\[ \hat{H} \Psi = E\Psi, \quad \hat{S}, \tilde{S} \]
\item I am not entirely happy about the accent placements, but it seems hard to solve right now \ldots
\item An integral equation:
  \[ u(x) + \lambda \int_0^1 k(x,y) u(y) \; dy = f(x), \quad x \in [0,1] \]
\item Inline integral: $w^{pq}_{rs} := \iint \phi_p(\mathbf{x})\phi_q(\mathbf{y})w(|\mathbf{x}-\mathbf{y}|) \phi_r(\mathbf{x})\phi_s(\mathbf{y})\; d\mathbf{x}d\mathbf{y}$
\item Display style integral:
\[ w^{pq}_{rs} := \iint \phi_p(\mathbf{x})\phi_q(\mathbf{y})w(|\mathbf{x}-\mathbf{y}|) \phi_r(\mathbf{x})\phi_s(\mathbf{y})\; d\mathbf{x}d\mathbf{y} \]
\item Some blackboard symbols: $\mathbb{C}$,$\mathbb{F}$,$\mathbb{R}$
\item A useful formula everyone should know:
\[ x_{\pm} = -\frac{b}{2} \pm \sqrt{\left(\frac{b}{2}\right)^2 - c} \]
\end{itemize}

\end{frame}


\begin{frame}{Theorems}
This is how a theorem block looks like in this version:
\begin{theorem}[First fundamental theorem of calculus]
Lef $f$ be a continuous real-valued fnction on the interval $[a,b]$. Let $F$ be the function defined by
\[ F(x) = \int^x_a f(y) \; dy. \]
Then, $F$ is uniformly continuous on $[a,b]$, differentible on $(a,b)$, and
\[ F'(x) = f(x). \]
\end{theorem}
\end{frame}


%\framebackground{dark}
%\setbeamercolor{background canvas}{fg=hylleraas-blue2}

\section{Alternate backgrounds}

\begin{frame}{Alternate bakcgrounds and colors}
  \begin{itemize}
  \item
    In the hylleraas theme, we have defined alternate backgrounds with corresponding alternate
    color schemes. This can be used to accenuate certain points and to liven up a little.
  \item
    The alternate schemes are accessed by passing an option to the \texttt{frame} environment.
  \end{itemize}
\end{frame}


\begin{frame}[dark]
    \frametitle{A darker scheme}

    \begin{itemize}
      \item  This slide shows a darker background and lighter text.
      \item Note the changed logo
    \end{itemize}

    \begin{theorem}[Errors]
      There is always one more error.
    \end{theorem}

\end{frame}

\begin{frame}[dark]
  \frametitle{~}
  \begin{fancyquote}[Albert Einstein]
    This slide shows a fancy quote on an darker background.
  \end{fancyquote}
\end{frame}

\begin{frame}[image 1]

  \frametitle{Fancy quotes}
  \begin{fancyquote}[Unknown]
    This slide shows an image background. There are four predefined images.
  \end{fancyquote}
\end{frame}

\begin{frame}[image 2]
\frametitle{Fancy quote slide}
\begin{fancyquote}[T.S. Eliot]
This slide has subtler background.
\end{fancyquote}
\end{frame}

\begin{frame}[image 3]
\frametitle{Fancy quote slide}
\begin{fancyquote}[John Lennon]
I am the egg man, \\
they are the egg men \\
I am the walrus, \\
goo goo g'joob
\end{fancyquote}
\end{frame}

\begin{frame}[image 4]
\frametitle{Fancy quote slide}
\begin{fancyquote}[Paul A.M. Dirac]
Never trust random quotes you find on the internet while googling solutions to your \LaTeX~problems.
\end{fancyquote}
\end{frame}

\begin{frame}
  \frametitle{An example}
  \begin{example}[Ok]
  Yes
  \end{example}
\end{frame}

%\framebackground{light}



\end{document}
